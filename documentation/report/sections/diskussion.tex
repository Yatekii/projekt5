Obwohl Vorkenntnisse in Sachen PCB-Design vorhanden waren, erwies sich die Arbeit als überaus anspruchsvoll. Da beide Teammitglieder bisher eher im digitalen Bereich Schaltungen entwickelt haben, war das Arbeiten mit Opamps und signaltechnisch kritischen Bausteinen eher neu und vieles war anfangs unklar.

Vom Rechenen mit Dezibel in der praktischen Anwendung über Leitungsterminierung bis hin zum Bedienen und Verstehen des VNAs waren viele Dinge zu lernen. Anfangs erschien es als gar nicht so viel Arbeit, obwohl Prof. Hufschmid davor warnte, den Aufwand nicht zu unterschätzen. Doch dann waren da so viele kleine Dinge, die man verstehen muss und die auch schief gehen können.

Zwar ist eine hübsche Leiterplatte entstanden, welche in den gemessenen Bereichen gut funktioniert und auch für weitere Messungen dienlich sein wird, jedoch blieben einige Arbeiten die offen waren auch weiterhin offen. Es war leider zeitlich nicht mehr im Rahmen, ein Redesign der Leiterplatte zu machen.

Was jedoch äusserst erfreulich ist, ist der Umstand, dass keiner der ursprünglichen Fehler mehr vorhanden ist oder reproduziert werden konnte.

\textbf{Noah Hüsser:} Anfangs dachte ich, dass wir ja eigentlich nur die Vorarbeit in grösser nachbauen und dann ausmessen. Dies sah für mich nach einer sehr viel kleineren Arbeit als die Vorgängerarbeit aus. Es zeigte sich jedoch schnell, dass 'nachbauen' nicht so einfach ist und man trotzdem jedes Bauteil kontrollieren muss, Datenblätter zu wälzen hat um Fehler zu finden und dann die ganze Theorie zu den Messungen auch noch verstehen muss. Ich habe in dem Projekt extrem viel gelernt, nicht nur in Hinsicht auf Elektrotechnik, sondern auch was das Einschätzen eines Arbeitsaufwandes betrifft und dass Teamarbeit nicht immer ganz einfach ist.

\subsection*{Offene Arbeiten / weiteres Vorgehen}
Es fehlen noch einige Messungen zum Gesamtsystem:
\begin{itemize}
    \item Messung der Rauschleistung
    \item Messung mit ADC am Gesamtsystem
    \item Messung der Intermodulation
\end{itemize}

Des Weiteren muss der DAC noch einmal im Gesamtsystem verwendet werden. Dies blieb leider bisher aus, da dieser auf der ersten Leiterplatte als mögliches Problem angesehen wurde und nicht mehr bestückt wurde.
Nicht zu vergessen ist das Evaluieren ob tatsächlich ein ADC mit 12 bit benötigt wird oder ob auch eine kleinere Auflösung taugen würde.

Bevor an ein Redesign gegangen werden kann, sollten unbedingt diese Messungen gemacht werden. Es wäre ein wenig gewagt zu sagen, dass dies nur noch wenig Zeit kostet. Es war zeitlich kostspielig alle Teile des AFEs zum Laufen zu kriegen. Und bevor nicht alles korrekt lief, war es auch nicht sinnvoll erweiterte Messungen zu machen. Nun sollten diese aber vorerst einfach zu machen sein, ohne dass weitere Dinge an der Leiterplatte verändert werden müssen.

\subsection*{Danksagung}

Wir möchten uns bei Prof. Hufschmid bedanken, der uns das Projekt zugetraut hat und sich auch die Zeit genommen hat uns Dinge zu erklären, die Anfangs nicht so trivial waren. Er hat auch gut vermittelt als die Teamarbeit nicht so gut lief.

Dann möchten wir uns bei Stefan Muhr bedanken, der immer da war im Labor, es an nichts mangeln liess und immer gute Laune dabei hatte.

Nicht vergessen werden soll Prof. Niklaus, der ungefragt im Labor vorbeikam und sich sehr viel Zeit nahm, um das Arbeiten mit dem VNA und den Messgeräten zu erklären und erleichtern.

Besten Dank!