Aus dem Vorgängerprojekt ist ein funktionsfähiger Prototyp eines SDR-Empfängers entstanden. Im Verlaufe des Projektes wurden jedoch mehrere Schwachpunkte und noch durchzuführende Arbeiten identifiziert. Viele davon entfallen auf den digitalen Teil. Die folgenden Punkte wurden zur Verbesserung im analogen Teil aufgelistet:

\begin{itemize}
	\item Anpassungen der Dimensionierung mehrerer passiver Komponenten
	\item Es fehlt ein Linearregler als rauscharme Stromversorgung
	\item Es fehlt ein Digital-Analog-Wandlers zum Einstellen der Verstärkung.
	\item Die Massenfläche sollte neu entworfen werden.
	\item Auf Signalleitungen treten Spannungsspitzen auf.
	\item Bei Frequenzen unter 2 MHz ist die Verstärkung deutlich geringer als erwartet.
	\item Die Verstärkung des einstellbaren Verstärkers entspricht nicht dem erwarteten Wert sondern hat einen Versatz
\end{itemize}
