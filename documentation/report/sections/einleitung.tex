Im Projekt 'Software Defined Radio mit FPGA' von Jonas Walter und Patrick Studer\cite{SDRprev} wurde ein SDR-Empänger auf Basis eines FPGA-Entwicklerboards aufgebaut. Dazu wurde im Verlauf des Projektes ein analoges Frontend, sowie Software für das FPGA entwickelt. Das Projekt war erfolgreich und konnte beim Abschluss einen funktionsfähigen Prototypen vorweisen.

Solche digitalen Empfänger haben grosse Vorteile, wie zum Beispiel die Auslagerung von Signalverarbeitung in den digitalen Bereich. Dies bringt hohe Flexibilität und einen im Vergleich kleinen Aufwand zur Funktionserweiterung mit sich.

Dieses Projekt versteht sich als Fortsetzung des 'Software Defined Radio mit FPGA'-Projekts und möchte bestimmte Bereiche des existierenden Prototypen überarbeiten. Insbesondere die Filterung und die Verstärkung des Eingangssignals sollen untersucht werden. Diese zeigten ein nicht vollends zufriedenstellendes Verhalten und sollen anhand von Fachliteratur evaluiert und wo möglich Ideen zur Verbesserung erdacht werden. Im Anschluss sollen die Vorschläge in einem neuen Prototypen umgesetzt werden. Zum Schluss sollen die Änderungen getestet und die Messwerte evaluiert werden.

Aus zeitlichen Gründen ist es nicht möglich den ganzen Aufbau im Detail zu untersuchen. Das Projekt beschränkt sich daher auf den analogen Teil dieser Schaltung. Der digitale Teil des SDR-Empfängers wird nicht weiter betrachtet und könnte in einem weiteren Projekt weiterentwickelt werden.

In diesem Bericht werden zunächst die grundlegenden theoretischen Prinzipien betrachtet und die spezifische Ausgangslage des existierenden analogen Frontends analysiert.
Im Anschluss wird der Aufbau des bestehenden Empfängers, sowie seine Schwachpunkte beschrieben und dann entsprechende Massnahmen zur Verbesserung vorgestellt. Darauffolgend werden die erfolgten Messungen an der umgesetzten Schaltung erläutert. Natürlich werden die Messresultate besprochen und Empfehlungen für weitere Entwicklungen ausgeführt. Abschliessend werden die Resultate kritisch betrachtet. Dabei stehen die Vorteile und Probleme der umgesetzten Lösung und mögliche Verbesserungen im Mittelpunkt.
